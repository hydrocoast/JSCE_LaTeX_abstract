%%%%%%%%%%%%%%%%%%%%%%%%%%%%%%%%%%%%%%%%%%%%%%%%%%%%%%%%%%%%%%%%%%%%%%%
% 2006 海講 アブストラクトサンプル
% 異常波浪の発生に及ぼす非線型干渉とスペクトル形状の影響
%%%%%%%%%%%%%%%%%%%%%%%%%%%%%%%%%%%%%%%%%%%%%%%%%%%%%%%%%%%%%%%%%%%%%%%
\documentclass[11pt,dvipdfmx]{jarticle}
\usepackage{kaikou_abstract2006}
\usepackage{graphicx}
%====================================================================
\begin{document}
\setlength{\baselineskip}{5.2mm}
%====================================================================
\Title{テスト}
%----------------------------------------------------------------------
\Section{緒言}
%----------------------------------------------------------------------
%\\
%1234567890
%1234567890
%1234567890
%1234567890
%1234567890
90年代初頭より,Freak waveの出現に高次の非線型相互作用の影響が大きいとの
認識が高まり,これと並行して水面変位の高次モ−メントと異常波浪との関係に
ついての研究が行われてきた.
その結果,3次の非線型干渉が強くなると,高次モ−メントの値が増大し,Freak
waveのような波形を持つ異常波の出現確率も高まることが明らかにされている.
しかし,力学的に求まる非線型干渉の強さと,統計量である高次モ−メントと
の直接的な関係が不明であり,統一的な理論が必要とされている.
本研究では,Zakharov方程式をもとに
高次の非線型干渉の強さと水面変位の高次モ−メントおよびスペクトル形状の関係について明らかにする.

%----------------------------------------------------------------------
\Section{研究の概要}
%----------------------------------------------------------------------
深海域を伝播する不規則波を対象とし,水面変位のアクションに関するZakharov
方程式(Zakharov,1968)を均一かつ定常波浪場(
%\begin{equation}
$
\langle a_1 a^*_2 \rangle = \frac{\omega}{2g} N_1 
\delta(\ensuremath{\vec{k}}_1-\vec{k}_2),
%\;{\rm かつ}\;
\langle a_1 a_2 \rangle =0
$
)
%\label{p:2b}
%\end{equation}
に書き換える.
ここで$a$は水面変位のフ−リエ振幅,$N$はアクション密
度,$\omega$は角周波数,$g$は重力加速度である.
記号$<>$は,アンサンブル平均を意味する.
アンサンブル平均したZakharov方程式より,3次の非線型干渉による4波相互作
用項$T_{1,2,3,4}$ (Krasitskii,1993) と水面変位の高次モ−メントの関係を得る.
\begin{eqnarray}
 \kappa_{40}
  = \frac{<\eta^4>}{m_0^2} - 3 
  = \mu_4 - 3
=
  \frac{12}{g^2m_0^2} 
  \int
%  \d\vk_{1,2,3,4}
  \d k_{1,2,3,4}
  T_{1,2,3,4}\sqrt{\omega_1\omega_2\omega_3\omega_4}
  \,\delta_{1+2-3-4}
  R_r(\Delta\omega,t)
  N_1 N_2 N_3
  \label{eqn:kurtosis-03}
\end{eqnarray}
\begin{eqnarray}
 \kappa_{22}=\frac{\langle \eta^2 \zeta^2 \rangle}{m_0^2}-1
  = \frac{1}{3} \kappa_{40},
  \label{p:14}
\end{eqnarray}
ここで,
$\kappa_{40}$は水面変位$\eta$の4次キュムラント,
$\kappa_{22}$は水面変位$\eta$と包絡線$\zeta$との2次の結合キュムラント,
$\mu_4$はkurtosis,
$R_r$は非線型エネルギ−伝達関数,
$m_0$は$\eta$のRMS値である.
ついで,式(1)を挟帯スペクトルの仮定の下で,周波数スペクトル$E(\omega)$に
ついて書き換えると次式を得る.
\begin{eqnarray}
 \kappa_{40}=\frac{12gk_0^3}{m_0^2}{\cal P}\int{\rm d}\omega_1{\rm d}\omega_2
{\rm d}\omega_3
\sqrt{\frac{\omega_4}{\omega_1\omega_2\omega_3}}
 \frac{ E(\omega_1) E(\omega_2) E(\omega_3)}
 {\Delta \omega},
\label{eqn:12}
\end{eqnarray}
ここで${\cal P}$は積分定数である.
式(2)より,与えられた周波数スペクトル$E(\omega)$より,
$\kappa_{40}$および$\kappa_{22}$を求めることができる.
さらに,$\kappa_{40}$とJanssen(2003)の提案した異常波浪指標であるBenjamin
Feir Index (BFI)の関係について調べ,スペクトル幅と不安定性および異常波浪
の関係を明かにした.
ついで,式(1)の関係を用いて,スペクトル形状と水面変位の4次モ−メント
$\mu_4(=\kappa_{40}-3)$とBFIの関係について調べた.
まず始めに,
周波数スペクトルにWallops型とJONSWAP,方向分布関数に光易型方向関数を与え,
$H_{1/3}$,$T_{1/3}$およびスペクトル形状パラメ−タの$m$および$\gamma$お
よび$S_{max}$を変化させて,$\mu_4$とBFIの変化を調べた.
図1および2に示すのはその一例であり,$S_{max}=45$,\ $T_{1/3}$=10sの条件下で式(1)を数値積分した結果である.
これより$T_{1/3}$一定の下では,$H_{1/3}$が
大きくなると$\mu_4$とBFIの値は単調に
増加することがわかった.
また周波数スペクトル幅が狭くなると($m\rightarrow100$, $\gamma\rightarrow10$)
$\mu_4$とBFIは急激に増加し,
3次の非線型干渉の影響が顕著に現れることがわかった.
最後に,非線型干渉の寄与とスペクトル幅の関係について解析的に調べ,
不規則波の不安定性限界とスペクトル幅の関係について定式化を行った.
%----------------------------------------------------------------------
\Section{結語}
%----------------------------------------------------------------------
以上,本研究では
これまで直接的関係が不明であった3次の非線型干渉と水面変位
の高次モ−メントの関係を定式化し,スペクトル形状と異常波浪の関係につ
いて明らかにした.

\begin{figure}
\includegraphics[width=15cm]{test_fig.eps}
\end{figure}


\end{document}
